\chapter{  }
\label{chap:appendix1}
\section{Sample code to identify failure domains}
\label{sec:appendix1}
\scriptsize


\textbf{Program 1} Program generated by ADFD on finding fault in SUT
\begin{lstlisting}
/**
 * Dynamically generated code by ADFD strategy 
 * after a fault is found in the SUT.
 * @author (Mian and Manuel)
 */
import java.io.*;
import java.util.*;

public class C0 
{
	public static ArrayList<Integer> pass = new ArrayList<Integer>();
	public static ArrayList<Integer> fail = new ArrayList<Integer>();
	public static boolean startedByFailing = false;
	public static boolean isCurrentlyFailing = false;
	public static int start = -80; 
	public static int stop = 80;

	public static void main(String []argv){
		checkStartAndStopValue(start);
		for (int i=start+1;i<stop;i++){
			try{
				PointDomainOneArgument.pointErrors(i);
				if (isCurrentlyFailing) 
				{
					fail.add(i-1);
					fail.add(0);
					pass.add(i);
					pass.add(0);
					isCurrentlyFailing=false; 
				} 
			} 
			catch(Throwable t) { 
				if (!isCurrentlyFailing) 
				{
					pass.add(i-1);
					pass.add(0);
					fail.add(i);
					fail.add(0);
					isCurrentlyFailing = true;
				}  
			} 
		} 
		checkStartAndStopValue(stop); 
		printRangeFail(); 
		printRangePass();  
	}

	public static void printRangeFail() { 
		try { 
			File fw = new File("Fail.txt"); 
			if (fw.exists() == false) { 
				fw.createNewFile(); 
			}
			PrintWriter pw = new PrintWriter(new FileWriter (fw, true));   
			for (Integer i1 : fail) { 
				pw.append(i1+"\n"); 
			} 
			pw.close(); 
		} 
		catch(Exception e) { 
			System.err.println(" Error : e.getMessage() "); 
		} 
	} 
	public static void printRangePass() { 
		try { 
			File fw1 = new File("Pass.txt"); 
			if (fw1.exists() == false) { 
				fw1.createNewFile(); 
			}
			PrintWriter pw1 = new PrintWriter(new FileWriter (fw1, true));   
			for (Integer i2 : pass) { 
				pw1.append(i2+"\n");
			} 
			pw1.close(); 
		} 
		catch(Exception e) { 
			System.err.println(" Error : e.getMessage() "); 
		} 
	} 
	public static void checkStartAndStopValue(int i) { 
		try { 
			PointDomainOneArgument.pointErrors(i);
			pass.add(i); 
			pass.add(0);
		} 
		catch (Throwable t) { 
			startedByFailing = true; 
			isCurrentlyFailing = true; 
			fail.add(i); 
			fail.add(0);
		} 
	} 
}

\end{lstlisting}


%%%%%%%%%%%%%%%%%%%%%%%%%%%%%%%%%%%%%%%%%%%%%%%%%%%%%%%%%%%%%%%%%%%%%%%%%%%%%%%%%%%%%%%%%%%%%%
\textbf{Program 2} Point domain with One argument
\begin{lstlisting} 
/**
 * Point Fault Domain example for one argument
 * @author (Mian and Manuel)
 */
public class PointDomainOneArgument{

	public static void pointErrors (int x){
		if (x == -66 )
			x = 5/0;

		if (x == -2 )
			x = 5/0;

		if (x == 51 )
			x = 5/0;

		if (x == 23 )
			x = 5/0;
	}
}
\end{lstlisting}
%%%%%%%%%%%%%%%%%%%%%%%%%%%%%%%%%%%%%%%%%%%%%%%%%%%%%%%%%%%%%%%%%%%%%%%%%%%%%%%%%%%%%%%%%%%%%%
\textbf{Program 3} Point domain with two argument
\begin{lstlisting}
/**
 * Point Fault Domain example for two arguments
 * @author (Mian and Manuel)
 */
public class PointDomainOneArgument{

	public static void pointErrors (int x, int y){
		int z = x/y;
	}

}
\end{lstlisting}

%%%%%%%%%%%%%%%%%%%%%%%%%%%%%%%%%%%%%%%%%%%%%%%%%%%%%%%%%%%%%%%%%%%%%%%%%%%%%%%%%%%%%%%%%%%%%%
\textbf{Program 4} Block domain with one argument
\begin{lstlisting}
/**
 * Block Fault Domain example for one arguments
 * @author (Mian and Manuel)
 */

public class BlockDomainOneArgument{

public static void blockErrors (int x){
	
	if((x > -2) \&\& (x < 2))
		x = 5/0;
	
	if((x > -30) \&\& (x < -25))
		x = 5/0;
	
	if((x > 50) \&\& (x < 55))
		x = 5/0;

   }
}

\end{lstlisting}
%%%%%%%%%%%%%%%%%%%%%%%%%%%%%%%%%%%%%%%%%%%%%%%%%%%%%%%%%%%%%%%%%%%%%%%%%%%%%%%%%%%%%%%%%%%%%%
\textbf{Program 5} Block domain with two argument
\begin{lstlisting}
/**
 * Block Fault Domain example for two arguments
 * @author (Mian and Manuel)
 */
public class BlockDomainTwoArgument{

	public static void pointErrors (int x, int y){

		if(((x > 0)&&(x < 20)) || ((y > 0) && (y < 20))){
		x = 5/0;
		}
  	
	}

}
\end{lstlisting}
%%%%%%%%%%%%%%%%%%%%%%%%%%%%%%%%%%%%%%%%%%%%%%%%%%%%%%%%%%%%%%%%%%%%%%%%%%%%%%%%%%%%%%%%%%%%%%

\textbf{Program 6} Strip domain with One argument
\begin{lstlisting}
/**
 * Strip Fault Domain example for one argument
 * @author (Mian and Manuel)
 */

public class StripDomainOneArgument{

	public static void stripErrors (int x){
	
		if((x > -5) && (x < 35))
			x = 5/0;
  	 }
}
\end{lstlisting}
%%%%%%%%%%%%%%%%%%%%%%%%%%%%%%%%%%%%%%%%%%%%%%%%%%%%%%%%%%%%%%%%%%%%%%%%%%%%%%%%%%%%%%%%%%%%%%
\textbf{Program 7} Strip domain with two argument
\begin{lstlisting}
/**
 * Strip Fault Domain example for two arguments
 * @author (Mian and Manuel)
 */
public class StripDomainTwoArgument{

	public static void pointErrors (int x, int y){

		if(((x > 0)&&(x < 40)) || ((y > 0) && (y < 40))){
		x = 5/0;
		}
  	
	}

}

\end{lstlisting}
%%%%%%%%%%%%%%%%%%%%%%%%%%%%%%%%%%%%%%%%%%%%%%%%%%%%%%%%%%%%%%%%%%%%%%%%%%%%%%%%%%%%%%%%%%%%%%
