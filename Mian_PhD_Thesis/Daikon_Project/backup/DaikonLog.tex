\documentclass{article}
\usepackage{graphicx}
\usepackage{hyperref}
\usepackage{verbatim}
\hypersetup{
    colorlinks,
    citecolor=black,
    filecolor=black,
    linkcolor=black,
    urlcolor=black
}



\begin{document}
\tableofcontents\title{Daikon PhD log}
\author{Mian Asbat Ahmad}

\maketitle


\section{Setting up Daikon}
To setup Daikon in ubuntu 12.10 
\begin{enumerate}
\item Go to user home directory $ \$~cd$
\item Create a folder $\$~mkdir~daikonparent $
\item Change Directory $\$~cd~daikonparent$
\item To get Daikon $\$~wget~http://pag.csail.mit.edu/daikon/download/daikon.tar.gz$
\item Untar the tar file $\$~tar~zxf~daikon.tar.gz$
\item This creates a $�daikonparent/daikon/�$ subdirectory.
\item To setup the classpath for Daikon
\item Go to user home directory $\$~cd$
\item open .bashrc file $\$~vi~.bashrc$
\item Write the following lines at the end of the file
\begin{enumerate}
\item $export~JAVA\_HOME=``/usr/lib/jvm/java-6-openjdk-amd64" $
\item $export~CLASSPATH=.:/home/mian/daikonparent/daikon  $
\item $export~CLASSPATH=\$CLASSPATH:/home/mian/daikonparent/daikon/daikon.jar$ 
\item $export~CLASSPATH=\$CLASSPATH:/home/mian/daikonparent/daikon/bin/daikon.bashrc $
\item Write changes and quit escape $:wq$
\end{enumerate}
\item Close the current terminal $\$~exit$
\item Open a new terminal and all set
\end{enumerate}

\subsection{Common Error}
If you get exception in thread main $java.lang.verifyerror~stackmaptable$, it is due to the latest verion of JDK. To get rid of this error uninstall the latest jdk and install the 1.6 verion. Daikon is not compatible with Open-JDK 7
\subsubsection{To remove versions of java}
Remove all the Java related packages (Sun, Oracle, OpenJDK, IcedTea plugins, GIJ):

\begin{enumerate}
\item $sudo~apt-get~update$
\item $apt-cache~search~java~|~awk~'{print(\$1)}'~|~grep~-E~-e~'\hat{}(ia32-)?(sun|oracle)-java'~-e~'\hat{}openjdk-'~-e~'\hat{}icedtea'~-e~'\hat{}(default|gcj)-j(re|dk)'~-e~'\hat{}gcj-(.*)-j(re|dk)'~-e~'java-common'~|~xargs~sudo~apt-get~-y~remove$
\item $sudo~apt-get~-y~autoremove$
\item Purge config files: $dpkg~-l~|~grep~\hat{}rc~|~awk~'{print(\$2)}'~|~xargs~sudo~apt-get~-y~purge$
\item Remove Java config and cache directory: $sudo~bash~-c~'ls -d~/home/*/.java'~|~xargs~sudo~rm~-rf$
\item Remove manually installed JVMs: $sudo~rm~-rf~/usr/lib/jvm/*$
\item Remove Java entries, if there is still any, from the alternatives: $ for~g~in~ControlPanel~java~java_vm javaws~jcontrol~jexec~keytool~mozilla-javaplugin.so~orbd~pack200~policytool~rmid~rmiregistry~servertool~tnameserv~unpack200~appletviewer~apt~extcheck~HtmlConverter~idlj~jar~jarsigner~javac~javadoc~javah~javap~jconsole~jdb~jhat~jinfo~jmap~jps~jrunscript~jsadebugd~jstack~jstat~jstatd~native2ascii~rmic~schemagen~serialver~wsgen~wsimport~xjc~xulrunner-1.9-javaplugin.so;~do~sudo~update-alternatives~--remove-all~\$g;~done$

\item Search for possible remaining Java directories:  $~sudo~updatedb~sudo~locate~-b~'\\pack200' $

\item If the command above produces any output like $/path/to/jre1.6.0_34/bin/pack200 $ remove the directory that is parent of bin, like this: $~sudo~rm~-rf~/path/to/jre1.6.0_34.$

\item For more details, Follow the second option on the following page
http://askubuntu.com/questions/84483/how-to-completely-uninstall-java
\end{enumerate}

\subsubsection{Re/Install Java}
\begin{enumerate}
\item apt-cache search openjdk
\item sudo apt-get install openjdk-6-jdk
\item \$java -version
\item http://www.mkyong.com/java/how-to-install-java-jdk-on-ubuntu-linux/
\end{enumerate}
\subsection{Testing Daikon Setup}
\begin{enumerate}
\item Go to StackAr folder \$ cd daikonparent/daikon/examples/java-examples/StackAr/DataStructures
\item Compile all java files with -g option \$ javac -g *.java
\item Go to previous folder \$ cd..
\item Run the combined command or seperate one by one as in the following.
\begin{enumerate}
\item Step 1: java daikon.Chicory --daikon DataStructures.StackArTester
\item Step 2: java daikon.Daikon DataStructures.StackArTester 
\item Combined: java daikon.Chicory DataStructures.StackArTester 
\end{enumerate}
\item The above commands will generate two files, if configured properly
\begin{enumerate}
\item StackArTester.dtrace.gz 
\item StackArTester.inv.gz 
\end{enumerate}
\item Daikon is now correctly Configured.
\end{enumerate}

\section{Setup, Configuration and Testing of RANDOOP}
\begin{enumerate}
\item Instruction manual for RANDOOP is available at http://randoop.googlecode.com/hg/doc/index.html 
\item There are three mode of getting RANDOOP, i.e. Easiest, easy and all. We will try easy one for now which contain the RANDOOP.jar and its source code.
\end{enumerate}

\subsection{Getting RANDOOP}
\begin{enumerate}
\item Go to user home directory \$ cd
\item Create a directory \$ mkdir randoopparent
\item Change directory \$ cd randoop
\item Get Randoop easy version \$ wget https://randoop.googlecode.com/files/randoop.1.3.3.zip
\item Unzip the file \$ unzip randoop.1.3.3.zip
\item Setup the classpath
\item Go to user home directory \$ cd
\item Open .bashrc file \$ vi .bashrc
\item Write the following lines at the end
\begin{enumerate}
\item export CLASSPATH=\$CLASSPATH:/home/mian/randoopparent/randoop 
\item export CLASSPATH=\$CLASSPATH:/home/mian/randoopparent/randoop/randoop.jar 
\item save and quite the file escape :wq
\item Running Randoop
\end{enumerate}
\item You can run randoop by the following command 
\item java -ea randoop.main.Main gentests --testclass=java.util.TreeSet --timelimit=60  The above command will start randoop for 60 seconds on the class TreeSet and the test cases will be saved in user home directory after execution.
\end{enumerate}
\section{Eclipse}

\subsection{Getting and Installing Eclipse}
\begin{enumerate}
\item Go to home directory \$ cd
\item $wget http://www.eclipse.org/downloads/download.php?file=/technology/epp/downloads/release/kepler/R/eclipse-standard-kepler-R-linux-gtk-x86_64.tar.gz$
\item tar -zxvf eclipse*.tgz
\item If you download it through browser then go to \$ cd ~Downloads
\item Copy the untar folder to home directory \$ cp -avr eclipse ../
\end{enumerate}
\subsection{Common Error}
Eclipse must be downloaded according to the JVM architecture. E.g if JVM is 64bit then Eclipse must be 64bit, if 32 bit then 32bit. 

\subsection{Get Eclipse shortcut to Ubuntu taskbar}
\begin{enumerate}
\item First, create a .desktop file to eclipse: $ gedit ~/.local/share/applications/opt_eclipse.desktop $
\item Then, paste this inside (dont forget to edit Exec and Icon values):

[Desktop Entry] \\*
Type=Application \\*
Name=Eclipse \\*
Comment=Eclipse Integrated Development Environment\\*
Icon=/home/mian/eclipse/icon.xpm\\*
Exec=/home/mian/eclipse/eclipse \\*
Terminal=false \\*
Categories=Development;IDE;Java; \\*
StartupWMClass=Eclipse \\*

\item To give execute permissions: $ chmod +x ~/.local/share/applications/opt_eclipse.desktop $
\item Finally drop opt\_eclipse.desktop to launcher nautilus $ ~/.local/share/applications $
\item For more details click [http://askubuntu.com/questions/80013/how-to-pin-eclipse-to-the-unity-launcher here]
\end{enumerate}

\begin{comment}
== Installing WikiMedia ==
* You must have webserver and mysql running
* To start webserver $ sudo service apache2 restart
* Go to http://www.mediawiki.org/wiki/Download
* Download mediawiki software
* Go to your download folder $ cd /home/mian/Downloads
* Unzip the file $ unzip mediawiki
* Copy the file to /var/www/ $ cp -a mediawiki /var/www
* Now open your internet browser and write in address bar www.localhost/mediawiki
* Follow the step by step procedure and on completion a file will be downloaded in /home/mian/Downloads folder
* copy the file and paste it into /var/www/mediawiki folder $ cp /home/mian/Downloads/Localsettings /var/www/mediawiki
* In the browser go to the same page again localhost/mediawiki and your mediawiki is up and running
* Start writing your logs

== Wiki Logo ==
Follow the following steps to change the logo <br>
* Get wikimedia 135 x 135 icon or get one from [http://www.inmotionhosting.com/support/edu/mediawiki/change-media-wiki-appearance/upload-custom-logo-mediawiki here]
* Save it to /var/www/mediawiki/images folder
* Open terminal and go to cd /var/www/mediawiki
* Open the file sudo vi LocalSettings.php
* Scroll down and change value to the path  $wgLogo  = "/mediawiki/images/wiki.png";
* Refresh the website and the logo will be there

\end{comment}

\end{document}
Go to home page  