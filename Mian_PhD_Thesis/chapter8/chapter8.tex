\externaldocument{chapter8}
\chapter{Future Work}
\label{chap:futureWork}

\section{Introduction}\label{sec:intro8}
	
This chapter presents the scope and potential of future work as an extension of the present research study. The topics suggested include: use of contracts and assertions to discover failure; introducing object distance in DSSR strategy to enhance its testing ability; measuring code coverage of the three newly developed strategies to assess additional aspect of performance; extension of ADFD+ strategy for testing non numerical data; enhancing the plotting ability of ADFD+ strategy to more than two-dimensional charts; introducing additional features in the user interface of ADFD+; extension of ADFD+ to apply it to the real world scenario; research on the frequency of failures laying in point block and strip domains; improvement in the DSSR strategy to reduce overhead. 

	
%While this research has demonstrated some of the potential of the newly developed automated random strategies, many opportunities for extending the work exist.
%This chapter presents some potential directions that such extensions could take. These include improving the implementation of DSSR to reduce overhead; Use of contracts and assertions to find a failure; introducing object distance in DSSR suggested by Ciupa et al. \cite{ciupa2006object}; measure of the coverage achieved by DSSR, ADFD and ADFD+; extend tool to test non-numerical and reference type data in ADFD+;
%adding the feature of plotting more than two dimensional charts; 
%improving the user interface of ADFD+; 
%evaluating the effectiveness of ADFD+ on real world applications; and characterizing the number of programs containing point, block and strip failure-domains.\\



\textbf{Use of contracts and assertions to discover failure}
%No explicit oracles were defined in our experiments to keep the process fully automated. 
The common practice to use undefined run-time exceptions of the programming language as test oracles in the absence of contracts and assertions was followed in the study. It is worthwhile to study the fault-detection ability of an automated strategy in the presence of contracts and assertions. To generate explicit oracles a tool like Daikon may be integrated in the system for achieving the automatic generation of invariants and their annotation in to source code.\\

\textbf{introducing object distance in DSSR strategy to enhance its testing ability}

The newly developed DSSR strategy add the neighbouring values for primitive type data and Strings. It has a limitation that no neighbouring values are added when the failure is found by  a reference type data. It is suggested for future research work to extend the DSSR strategy by incorporating a suitable technique like Artoo to find the neighbouring objects for including these in the list of interesting values.  \\


\textbf{measuring code coverage of the three newly developed strategies to assess additional aspect of performance}
In spite of the fact that the strategies developed in the study generate more test cases from the surrounding area where a failure is discovered for better coverage, it is worthwhile to measure the code coverage achieved by the new strategies to ensure the effectiveness. The instrumentation technique may be applied to the software under test to achieve the desired objective. \\
%We predict that the techniques developed in the study will achieve better coverage because they generate more test cases from the area where one failure is discovered. The discovery of failure identifies a legitimate value in most of the cases. However, Random strategies typically achieve low level of coverage~\cite{oriol2010yeti} and all the techniques are originated from random strategy. Therefore, it would be interesting to measure the coverage achieved by DSSR, ADFD and ADFD+.\\

\textbf{Extend ADFD and ADFD+ to test non-numerical and reference type data}

Current implementation of ADFD and ADFD+ tests only numerical programs. This restricts the usability of ADFD+ for production software of non-numerical data types. This can be solved by extending the tool to include testing of other primitive and reference data types. \\

\textbf{Extend ADFD and ADFD+ to represent multi-dimensional programs}

The ADFD and ADFD+ in its present state is capable of graphical representation of only one and two dimensional numerical programs. While it would be difficult, it is definitely worthwhile to graphically represent multi dimensional numerical and non numerical programs.\\

\textbf{Improving the user interface of ADFD+}

The GUI of ADFD+ provides a fully automated way of evaluating the program under test and representing the results in a graphical form. However the GUI can be extended to give the tester more control (like real time interaction and manual addition of test cases) and better sight of the output graph. More views can also be added like thumbnail view to show more graphs on a single screen and 3D support to present multi-dimensional arguments.\\

\textbf{Evaluating the effectiveness of ADFD+ in real world programs}

The experiments performed to assess the effectiveness of ADFD and ADFD+ used error-seeded programs. This generate a threat to validity i.e. to what degree the classes under test are representatives of true practice. The threat may be reduced to a greater extent in future experiments by taking programs containing both numerical and non-numerical data.  \\

\textbf{Characterizing the number of programs containing point, block and strip failure-domains}

While we know from the literature that the point, block and strip domain of failure exist in the input domain. There is no concrete study to verify how frequent they are and which failure-domain exist more in number than the other two. Its identification can help testers to apply the most appropriate testing technique to the SUT becuase we know that random testing performs better when errors forms block and strip domains as against point domains.\\
	




