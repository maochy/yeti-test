\chapter{Invariant Guided Random+ Strategy}
\label{chap:IGRS}

\section{Introduction}\label{sec:intro6}

\section{Invariant Guided Random+ Strategy}\label{sec:igrs}

\subsection{Daikon}

\subsection{Random Plus Strategy (R+)}
The random+ strategy~\cite{Leitner2007} is an extension of the random strategy. It uses some special pre-defined values which can be simple boundary values or values that have high tendency of finding faults in the SUT. Boundary values~\cite{Beizer1990} are the values on the start and end of a particular type. For instance, such values for \verb+int+ could be \verb+MAX_INT+, \verb+MAX_INT-1+, \verb+MAX_INT-2+; \verb+MIN_INT+, \verb-MIN_INT+1-, \verb-MIN_INT+2-. Similarly, the tester might also add some other special values that he considers effective in finding faults for the SUT. For example, if a program under test has a loop from -50 to 50 then the tester can add -55 to -45, -5 to 5 and 45 to 55 to the pre-defined list of special values. This static list of interesting values is manually updated before the start of the test and has slightly high priority than selection of random values because of more relevance and high chances of finding faults for the given SUT. These special values have high impact on the results, particularly for detecting problems in specifications~\cite{Ciupa2008}.


\subsection{Structure of the Invariant Guided Random+ Strategy}

%%%%%%%%%%%%%%%%%%%%%%%%%%%%% EXPLANATION OF IGRS STRATEGY %%%%%%%%%%%%%%%%%%%%%%%%%%


\subsection{Explanation of IGRS strategy on a concrete example}

%%%%%%%%%%%%%%%%%    IMPLEMENTATION OF IGRS STRATEGY   %%%%%%%%%%%%


\section{Implementation of the IGRS strategy} \label{sec:imp}


%%%%%%%%%%%%%%%%%    EVALUATION   %%%%%%%%%%%%%%%%%%%%


\section{Evaluation}\label{sec:eval}

\subsection{Research questions}
\begin{enumerate}
\item A
\item B
\item C
\end{enumerate}



\subsection{Experiments}


\subsection{Performance measurement criteria}

\section{Results}  \label{sec:res}
% How to write relative standard deviation.
% Eventually, the standard deviations are all of the order of magnitude of .1\% for all strategies.

\subsection{Answer A}

\subsection{Answer B}

\subsection{Answer C}




%%%%%%%%%%%%%%%%%    DISCUSSION   %%%%%%%%%%%%%%%%%%%%

\section{Discussion} \label{sec:discussion3}

%%%%%%%%%%%%%%%%%    RW   %%%%%%%%%%%%%%%%%%%%

\section{Related Work} \label{sec:rw}

%%%%%%%%%%%%%%%%%    CONCLUSIONS   %%%%%%%%%%%%%%%%%%%%

\section{Conclusions} \label{sec:conc}

