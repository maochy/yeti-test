In this chapter we give a brief introduction and motivation for the research work presented in this thesis. We commence by introducing the problems in random testing. We then describe the alternative approaches to overcome these problems, followed by our research goals and contributions. At the end of the chapter, we give an outline of the thesis.

\section{The Problems}
In software testing, one is often confronted with the problem of selecting a test data set, from a large or often infinite domain, as exhaustive testing is not always applicable. Test data set is a subset of domain carefully selected to test the given software. Finding an adequate test data set is a crucial process in any testing technique as it aims to represent the whole domain and evaluate the given system under test (SUT) for structural or functional properties \cite{mccabe1983} \cite{howden1986}. Manual test data set generation is a time-consuming and laborious exercise \cite{korel1990}, therefore, automated test data set generation is always preferred. Test data generators are classified in to Pathwise, Goal-Oriented, Intellighent and Random \cite{wiki2013}. Random test data generation generate test data set randomly from the whole domain. Unlike other approaches Random approach is simple, widely applicable, easy to implement in an automatic testing tool, fastest in computation, no overhead in choosing inputs and free from bias \cite{Ciupa2007}.  

Despite the benefits random testing offers, its simplistic and non-systematic nature expose it to high criticism \cite{white1987}. Myers \& Sandler \cite{Myers2004} mentioned it as ``Probably the poorest methodology of all is random-input testing...". Where this statement is based on intuition and lacks any experimental evidence, it motivated the interest of research community to evaluate and improve random testing. Adaptive random testing \cite{Chen2008}, Restricted Random Testing \cite{Chan2002}, Feedback directed Random Test Generation \cite{Pacheco2007a}, Mirror Adaptive Random Testing \cite{Chen2003} and Quasi Random Testing \cite{Chen2005} are few of the enhanced random testing techniques aiming to increase its fault finding ability. 

Random testing is also considered weak in providing high code coverage \cite{Offutt1996}, \cite{cohen1997}. For example, in random testing when the conditional statement  ``if (x == 25) then ... "  is exposed to execution then there is only one chance, of the ``then..." part of the statement, to be executed out of $2^\text{32}$. If x is an integer variable of 32 bit value \cite{Godefroid2005}. 

Random testing is no exception when it comes to the complexity of understanding and evaluating test results. Modern testing techniques simplifies results by truncating the lengthy log files and display only the fault revealing test cases in the form of unit tests. However efforts are required to show the test results in more compact and user friendly way. 


\section{Our Goals}
The overall goal of this thesis is to develop new techniques for automated testing based on random strategy that addresses the above mentioned problems. Particularly,

\begin{enumerate}
\item We aim to develop an automated random testing technique which is able to generate more fault-revealing test data. To achieve this we exploit the presence of fault clusters found in the form of block and strip fault domains inside the input domain of a given SUT. Thus we are able to find equal number of faults in fewer number of test cases than other random strategies.

\item We aim to develop a novel framework for finding the faults, their domains and the presentation of obtained results on a graphical chart inside the specified lower and upper bound. It considers the correlations of the fault and fault domain. It also gives a simplified and user friendly report to easily identify the faulty regions across the whole domain.

\item We aim to develop another automated random strategy that utilises the literals of the given SUT to increase code coverage and quickly identify semantically small faults. On test initialisation it scan binary of the given SUT for literals. These literals are added to the list of interesting values with high preference of usage as test data. Thus we are able to handle the problem of code coverage and quick identification of faults in random testing. 
\end{enumerate}

\section{Contributions}
To achieve the research goals described in Section xx, we make the following specific contributions:

\subsection{Dirt Spot Sweeping Random Strategy}
Development of a new, enhanced and improved form of automated random testing: the Dirt Spot Sweeping Random (DSSR) strategy. This strategy is based on the assumption that faults and unique failures reside in contiguous blocks and stripes. The DSSR strategy starts as a regular random+ testing strategy � a random testing technique with preference for boundary values. When a failure is found, it increases the chances of using neighbouring values of the failure in subsequent tests, thus slowly sweeping values around the failures found in hope of finding failures of different kind in its vicinity.
The DSSR strategy is implemented in the YETI random testing tool. It is evaluated against random (R) and random+ (R+) strategies by testing 60 classes (35,785 line of code) with one million (105) calls for each session, 30 times for each strategy. The results indicate that for 31 classes, all three strategies find the same unique failures. We analysed the 29 remaining classes using t-tests and found that for 7 classes DSSR is significantly better than both R+ and R, for 8 classes it performs similarly to R+ and is significantly better than R, and for 2 classes it performs similarly to ran- dom and is better than R+. In all other cases, DSSR, R+ and R do not perform significantly differently. Numerically, the DSSR strategy finds 43 more unique failures than R and 12 more unique failures than R+.

\subsection{Automated Discovery of Failure Domain}
There are several automated random strategies of software testing based on the presence of point, block and strip fault domains inside the whole input domain. As yet no particular, fully automated test strategy has been developed for the discovery and evaluation of the fault domains. We therefore have developed Automated Discovery of Failure Domain, a new random test strategy that finds the faults and the fault domains in a given system under test. It further provides visualisation of the identified pass and fail domain. In this paper we describe ADFD strategy, its implementation in YETI and illustrate its working with the help of an example. We report on experiments in which we tested error seeded one and two-dimensional numerical programs. Our experimental results show that for each SUT, ADFD strategy successfully performs identification of faults, fault domains and their representation on graphical chart.

\subsection{Random Plus Plus Strategy}
Acknowledgement of random testing being simple in implementation, quick in test case generation and free from any bias, motivated research community to do more for increase in performance, particularly, in code coverage and fault-finding ability. One such effort is Random+ --- Ordinary random testing technique with addition of interesting values (border values) of high preference. We took a step further in the same direction and developed Random++ strategy. Random++ strategy is an extended form of Random+ strategy with addition of program literals.  Literals from the given software under test are collected by instrumentation prior to testing and added to the list of interesting values.  Experimental result shows that Random++ not only increase the code coverage but also find some subtle errors that pure Random and Random+ were either unable or may take a long time to find.  



 
\section{Thesis Outline}

The rest of the thesis is organised as follows: In Chapter 2, we give a thorough review of the relevant literature. We commence by discussing a brief introduction of software testing and shed light on various techniques and types of software testing. Then, we extend our attention to automated random testing and the testing tools using random technique to test softwares. In Chapter 3, we present our first automated random strategy Dirt Spot Sweeping Random (DSSR) strategy based on sweeping faults from the clusters in the input domain. Chapter 4 describes our second automated random strategy which focus on dynamically finding the fault with their domains and its graphical representation. Chapter 5 presents the third strategy that focus on quick identification of faults and increase in coverage with the help of literals; Finally, in Chapter 7, we summarise the contributions of this thesis, discuss the weaknesses in the work, and suggest avenues for future work.






%Today, the primary focus of software companies is to achieve high quality. These companies spend an estimated thirty to ninety percent of the total software development cost on testing \ref{Beizer1990}, \ref{Standards2002}. In spite of spending 

%Software testing is the process of executing a software with specific test data followed by evaluation of the results to check whether it is working according to its specification or not \ref{Sommerville2006}.
% check here if we can replace specification with oracle or not.
%The test passes if the output complies to its specification and fails otherwise. The success of testing correlates with the number of failures found in the Software Under Test (SUT): a test is more successful if it finds more faults.

%It is interesting that program testing is used to show the presence of bugs, rather than absence of bugs [6]. Therefore the SUT that passes all the tests without returning a single failure does not guarantee that there is no fault. The testing process increases however the reliability and confidence of both the developers and the users in the tested product [7] [8] [9].

%Random testing is a black-box testing technique in which the SUT is executed against ran- domly selected test data. Test results obtained are compared either against the oracle defined, using SUT specifications in the form of assertions or exceptions defined by the programming language. The rapid increase in software development in today?s modern world prompts the need for automated testing to ensure high quality. The generation of random test data is com- paratively cheap and does not require too much intellectual and computation efforts [10] [11]. It is for this reason that various researchers have recommended this strategy for incorporation in automatic testing tools [12]. YETI [13] [14], AutoTest [15] [16], QuickCheck [17], Randoop [18], JArtage [19] are a few of the most common tools based on random strategy.
