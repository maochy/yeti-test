%%% Thesis Introduction --------------------------------------------------
\chapter{Introduction}
%\ifpdf
%    \graphicspath{{Introduction/IntroductionFigs/PNG/}{Introduction/IntroductionFigs/PDF/}{Introduction/IntroductionFigs/}}
%\else
%    \graphicspath{{Introduction/IntroductionFigs/EPS/}{Introduction/IntroductionFigs/}}
%\fi

This chapter includes motivation for the research study followed by the problems in random testing, the alternative approaches to overcome these problems, the research objectives and contributions of the study. At the end of the chapter, the structure of the remaining thesis is given.

\section{Motivation}
Software is a very important and essential component of computer system without which no task can be accomplished. Some softwares are developed for use in simple day to day operations while others are for highly complex processes in specialized fields like research and education, business and finance, defence and security, health and medicine, science and technology, aeronautics and astronautics, commerce and industry, information and communication, environment and safety etc. The margin of error in mission-critical and safety-critical systems is so small that a minor fault can lead to huge economic losses to the organization \cite{huang2004securing}. According to ...., estimated rate of private bugs have ranged from 15 to 150 bugs per 100 statements. 

Therefore, software companies leave no stone unturned to ensure the reliability and accuracy of the software. This dissertation is a humble contribution to the literature on the subject, with the aim to reduce the overall cost of software testing by devising new, improved and effective software testing techniques based on random strategy.

Random testing is a process in which generation of test data is random but according to requirements, specifications or any other test adequacy criteria. The given SUT is executed against the test data and results obtained are evaluated to determine whether the output produced satisfies the expected results.

\begin{figure}[h]
	\centering
		\includegraphics[width=14cm, height=6cm ]{Introduction/SoftwareTesting.png}
		\caption{Three main phases of random testing}
	\label{fig:SoftwareTesting}
\end{figure}

\section{The Problems}
Exhaustive testing of software is not always possible and the problem of selecting a test data set, from a large/infinite domain is often confronted. Test data set, as a subset of the whole domain, is carefully selected for testing the given software. Adequate test data set is a crucial factor in any testing technique because it represents the whole domain for evaluating the structural and/or functional properties \cite{howden1986}, \cite{mccabe1983}. Generating test data set manually is a time-consuming and laborious exercise \cite{korel1990}; Therefore, automated test data set generation is always preferred. Data generators can be of different types i.e. Path-wise, Goal-Oriented, Intelligent or Random \cite{wiki2013}. Random generator produces test data set randomly from the whole domain. Unlike other approaches random technique is simple, widely applicable, easy to implement, faster in computation, free from bias and costs minimum overhead \cite{Ciupa2007}.  According to Godefroid et al, ``Random testing is a simple and well-known technique which can be remarkably effective in discovering software bugs" \cite{Godefroid2005}.

Despite the benefits of random testing, its simplistic and non-systematic nature exposes it to high criticism \cite{white1987}. Myers \& Sandler \cite{Myers2004} mentioned it as, ``Probably the poorest methodology of all is random-input testing...". However, -------- reported that the above stated statement is based on intuition and lacks any experimental evidence [number]. The criticism motivated the researchers to look into various aspects of random testing for evaluation and possible improvement. Adaptive random testing (ART) \cite{Chen2008}, Restricted Random Testing (RRT) \cite{Chan2002}, Feedback Directed Random Testing (FDRT) \cite{Pacheco2007a}, Mirror Adaptive Random Testing (MART) \cite{Chen2003} and Quasi Random Testing (QRT) \cite{Chen2005} are a few of the enhanced random testing techniques reported in the literature.

Random testing is also considered weak in providing high code coverage \cite{cohen1997}, \cite{Offutt1996}. For example, in random testing when the conditional statement  ``{\it if (x == 25) then ... }"  is exposed to execution then there is only one chance, of the ``{\it then...}" part of the statement, to be executed out of $2^\text{32}$. If {\it x} is an integer variable of 32 bit value \cite{Godefroid2005}. 

Random testing is no exception when it comes to the complexity of understanding and evaluating test results. Modern testing techniques simplify results by truncating the lengthy log files and displaying only the fault revealing test cases in the form of unit tests. Further efforts are required to get the test results of random testing in more compact and user-friendly way. 


\section{Research Goals} \label{ResearchGoals}
The main goal of the research study is to develop new techniques for automated random testing with the aim to achieve the following objectives:

\begin{enumerate}
\item To develop a testing strategy with the capability to generate more fault-finding test data.

\item To develop a testing technique for finding faults, fault domains and presentation of results on a graphical chart within the specified lower and upper bound. 

\item To develop a testing framework with focus on increase in code coverage along with generation of more faultfinding test data. 

\end{enumerate}

\section{Contributions}
The main contributions of the thesis research are stated below: 

\subsection{Dirt Spot Sweeping Random Strategy}
%Development of a new enhanced and improved form of automated random testing: the Dirt Spot Sweeping Random (DSSR) strategy. This strategy is based on the assumption that faults and unique failures reside in contiguous blocks and stripes. The DSSR strategy starts as a regular random+ testing strategy � a random testing technique with preference for boundary values. When a failure is found, it increases the chances of using neighbouring values of the failure in subsequent tests, thus slowly sweeping values around the failures found in hope of finding failures of different kind in its vicinity.
%The DSSR strategy is implemented in the YETI random testing tool. It is evaluated against random (R) and random+ (R+) strategies by testing 60 classes (35,785 line of code) with one million ($10^\text{6}$) calls for each session, 30 times for each strategy. The results indicate that for 31 classes, all three strategies find the same unique failures. We analysed the 29 remaining classes using t-tests and found that for 7 classes DSSR is significantly better than both R+ and R, for 8 classes it performs similarly to R+ and is significantly better than R, and for 2 classes it performs similarly to random and is better than R+. In all other cases, DSSR, R+ and R do not perform significantly differently. Numerically, the DSSR strategy finds 43 more unique failures than R and 12 more unique failures than R+.

The faultfinding ability of the random testing technique decreases when the failures lie in contiguous locations across the input domain. To overcome the problem, a new automated technique: Dirt Spot Sweeping Random (DSSR) strategy was developed. It is based on the assumption that unique failures reside in contiguous blocks and stripes. When a failure is identified, the DSSR strategy selects neighboring values for the subsequent tests. Resultantly, selected values sweep around the failure, leading to the discovery of new failures in the vicinity. Results presented in Chapter \ref{chap:DSSR} indicated higher faultfinding ability of DSSR strategy as compared with Random (R) and Random+ (R+) strategies.

\subsection{Automated Discovery of Failure Domain}
The existing random strategies of software testing discover the faults in the SUT but lack the capability of locating the fault domains. In the current research study, a fully automated testing strategy named, ``Automated Discovery of Failure Domain (ADFD)" was developed with the ability to find the faults as well as the fault domains in a given SUT and provides visualization of the identified pass and fail domains in the form of a chart. The strategy is described, implemented in YETI, and practically illustrated by executing several programs of one and two dimensions in the Chapter \ref{chap:ADFD}. The experimental results proved that ADFD strategy automatically performed identification of faults and fault domains along with graphical representation in the form of chart.

\subsection{Invariant Guided Random+ Strategy}
Another random test strategy named, ``Invariant guided Random+ Strategy (IGR+S)�" was developed in the current research study. IGR+S is an extended form of Random+ strategy guided by software invariants. Invariants from the given SUT are collected by Daikon�--- an automated invariant detector for reporting likely invariants and adding them to the SUT as assertions. The IGR+S is implemented in YETI and generates values in compliance with the added assertions. Experimental results presented in Chapter \ref{chap:IGR+S} indicated improved features of IGR+S in terms of higher code coverage and identification of subtle errors that R, R+ and DSSR strategies were either unable to accomplish or required larger duration.  



 
\section{Structure of the Thesis}
%
The rest of the thesis is organized as follows: In Chapter 2, a thorough review of the relevant literature is given. It includes a brief introduction of software testing techniques followed by automated random testing tools. Chapter 3 describes Dirt Spot Sweeping Random (DSSR) strategy, which is based on sweeping of fault clusters in the input domain. Chapter 4 presents the newly developed Automated Discovery of Fault Domains  (ADFD) strategy, which focuses on dynamically finding the faults and domains along with their graphical representation. Chapter 5 presents the new strategy Invariant Guided Random+ Strategy (IGR+S) developed with the focus on quick identification of faults and increase in code coverage with the help of assertions. Chapter 6 summarizes contributions of the thesis research, discusses the strength and weaknesses of the study, gives conclusion and suggests avenues for future work. Chapter 7 ?






%Today, the primary focus of software companies is to achieve high quality. These companies spend an estimated thirty to ninety percent of the total software development cost on testing \ref{Beizer1990}, \ref{Standards2002}. In spite of spending 

%Software testing is the process of executing a software with specific test data followed by evaluation of the results to check whether it is working according to its specification or not \ref{Sommerville2006}.
% check here if we can replace specification with oracle or not.
%The test passes if the output complies to its specification and fails otherwise. The success of testing correlates with the number of failures found in the Software Under Test (SUT): a test is more successful if it finds more faults.

%It is interesting that program testing is used to show the presence of bugs, rather than absence of bugs [6]. Therefore the SUT that passes all the tests without returning a single failure does not guarantee that there is no fault. The testing process increases however the reliability and confidence of both the developers and the users in the tested product [7] [8] [9].

%Random testing is a black-box testing technique in which the SUT is executed against ran- domly selected test data. Test results obtained are compared either against the oracle defined, using SUT specifications in the form of assertions or exceptions defined by the programming language. The rapid increase in software development in today?s modern world prompts the need for automated testing to ensure high quality. The generation of random test data is com- paratively cheap and does not require too much intellectual and computation efforts [10] [11]. It is for this reason that various researchers have recommended this strategy for incorporation in automatic testing tools [12]. YETI [13] [14], AutoTest [15] [16], QuickCheck [17], Randoop [18], JArtage [19] are a few of the most common tools based on random strategy.


%%% ----------------------------------------------------------------------


%%% Local Variables: 
%%% mode: latex
%%% TeX-master: "../thesis"
%%% End: 
