\chapter{Conclusion}
\label{chap:conclusion}

\section{Introduction}\label{sec:intro7}



The thesis has introduced and assessed the efficiency's of three new automated strategies to perform random testing, namely Dirt Spot Sweeping Random (Chapter~\ref{chap:chapter4}), Automated Discovery of Failure Domain (Chapter~\ref{chap:chapter5} and Automated Discovery of Failure Domain+ (Chapter~\ref{chap:chapter6}). 

Section ~\ref{sec:chapter6} (page x) and Section ~\ref{sec:chapter6} (page x) presented the goals and contributions of the research, which are briefly recapped here.

The first strategy DSSR, tries to finds a failure and search for more failures around the discovered one by taking into consideration its boundary values. The second strategy, ADFD identifies failure and failure-domains in one and two-dimensional numerical programs and present them in a graphical report on X, Y chart. The third and final strategy, ADFD+ is an improvement of ADFD strategy with a better algorithm to find failure and failure-domains and its presentation in the graphical form. The ADFD+ output is compared with the Daikon's output to analyse their behaviour in response to known failure domains. It also give suggestions towards improving the Daikon algorithm with respect to failure-domains.




