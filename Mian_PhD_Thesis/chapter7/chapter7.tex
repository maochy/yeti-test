\chapter{Conclusion}
\label{chap:conclusion}

\section{Introduction}\label{sec:intro7}

Testing is fundamental requirement to assess the quality of any software. Manual testing is labour-intensive and error-prone; therefore emphasis is to use automated testing that significantly reduces the cost of software development process and its maintenance \cite{beizer1995black}. Most of the modern black-box testing techniques execute the System Under Test (SUT) with specific input and compare the obtained results against the test oracle. A report is generated at the end of each test session containing any discovered faults and the input values which triggers the faults. Debuggers fix the discovered faults in the SUT with the help of these reports. The revised version of the system is given back to the testers to find more faults and this process continues till the desired level of quality, set in test plan, is achieved.

The fact that exhaustive testing for any non-trivial program is impossible, compels the testers to come up with some strategy of input selection from the whole input domain. Pure random is one of the possible strategies widely used in automated tools. It is intuitively simple and easy to implement \cite{Ciupa2008},  \cite{Forrester2000}. It involves minimum or no overhead in input selection and lacks human bias \cite{hamlet1994},  \cite{Linger1993}. While pure random testing has many benefits, there are some limitations as well, including low code coverage \cite{Offutt1996} and discovery of lower number of faults \cite{Chen1994}. To overcome these limitations while keeping its benefits intact many researchers successfully refined pure random testing. Adaptive Random Testing (ART) is the most significant refinements of random testing. Experiments performed using ART showed up to 50\% better results compared to the traditional/pure random testing  \cite{Chen2008}.  Similarly Restricted Random Testing (RRT) \cite{Chan2002}, Mirror Adaptive Random Testing (MART)  \cite{Chen2004}, Adaptive Random Testing for Object Oriented Programs (ARTOO) \cite{Ciupa2008}, Directed Adaptive Random Testing (DART)  \cite{Godefroid2005}, Lattice-based Adaptive Random Testing (LART) \cite{Mayer2005} and Feedback-directed Random Testing (FRT) \cite{Pacheco2007} are some of the variations of random testing aiming to increase the overall performance of pure random testing.

All the above-mentioned variations in random testing are based on the observation of Chan et. al.,  \cite{Chan1996} that failure causing inputs across the whole input domain form certain kinds of domains. They classified these domains into point, block and strip fault domain. In Figure \ref{fig:patterns} the square box represents the whole input domain. The black point, block and strip area inside the box represent the faulty values while white area inside the box represent legitimate values for a specific system. They further suggested that the fault finding ability of testing could be improved by taking into consideration these failure domains.
All the above-mentioned variations in random testing are based on the observation of Chan et. al.,  \cite{Chan1996} that failure causing inputs across the whole input domain form certain kinds of domains. They classified these domains into point, block and strip fault domain. In Figure \ref{fig:patterns} the square box represents the whole input domain. The black point, block and strip area inside the box represent the faulty values while white area inside the box represent legitimate values for a specific system. They further suggested that the fault finding ability of testing could be improved by taking into consideration these failure domains.
All the above-mentioned variations in random testing are based on the observation of Chan et. al.,  \cite{Chan1996} that failure causing inputs across the whole input domain form certain kinds of domains. They classified these domains into point, block and strip fault domain. In Figure \ref{fig:patterns} the square box represents the whole input domain. The black point, block and strip area inside the box represent the faulty values while white area inside the box represent legitimate values for a specific system. They further suggested that the fault finding ability of testing could be improved by taking into consideration these failure domains.
All the above-mentioned variations in random testing are based on the observation of Chan et. al.,  \cite{Chan1996} that failure causing inputs across the whole input domain form certain kinds of domains. They classified these domains into point, block and strip fault domain. In Figure \ref{fig:patterns} the square box represents the whole input domain. The black point, block and strip area inside the box represent the faulty values while white area inside the box represent legitimate values for a specific system. They further suggested that the fault finding ability of testing could be improved by taking into consideration these failure domains.
All the above-mentioned variations in random testing are based on the observation of Chan et. al.,  \cite{Chan1996} that failure causing inputs across the whole input domain form certain kinds of domains. They classified these domains into point, block and strip fault domain. In Figure \ref{fig:patterns} the square box represents the whole input domain. The black point, block and strip area inside the box represent the faulty values while white area inside the box represent legitimate values for a specific system. They further suggested that the fault finding ability of testing could be improved by taking into consideration these failure domains.