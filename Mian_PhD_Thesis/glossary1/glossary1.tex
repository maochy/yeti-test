\begin{glossary1}

\textbf{Defect} When the distinction between fault and failure is not critical, defect can be used as a generic term to refer to either a fault (cause) or a failure (effect).

\textbf{Detection} Detection refers to the observation that the programs observed behavior differs from the expected behaviour.

\textbf{Effectiveness} The number of defects found by a technique.

\textbf{Efficiency} The amount of time taken by a technique to find a defect. In other words, it also specifies number of defects found by a technique in an hour.

\textbf{Error} An incorrect or missing human action that result in software containing a fault (i.e. incorrect software)

\textbf{Failure} An inability of a system to perform its required functions within specified requirements or to perform in an unexpected way.

\textbf{Fault} An abnormal condition that may cause a reduction in, or loss of, the capability of a functional unit to perform a required function. It can also be defined as a requirements, design, or implementation flaw or deviation from a desired or intended state.

\textbf{Isolation} Isolation means to reveal the root cause of the failure in the program.

\textbf{Test case} A set of inputs, execution conditions, and a pass/fail criterion.

\textbf{Test case specification} It is a requirement to be satisfied by one or more test cases.

\textbf{Test obligation} A partial test case specification, requiring some property deemed important to thorough testing.

\textbf{Test or test execution} The activity of executing test cases and evaluating their results.
\textbf{Test Plan} A document describing the estimation of the test efforts, approach, required resources and schedule of intended testing activities.

\textbf{Test Procedure} A document providing detailed instructions for the execution of one or more test cases.

\textbf{Test suite} A set of test cases.

\textbf{Testing Technique} Different methods of testing particular features a computer program, system or product. Testing techniques means what methods or ways would be applied or calculations would be done to test a particular feature of software.

\textbf{Validation} The process of evaluating the to determine whether it satisfies specified customer requirements.

\textbf{Verification} The process of evaluating the software to determine whether it works according to the specified requirements.

\end{glossary1}