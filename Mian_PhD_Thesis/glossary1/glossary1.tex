\begin{glossary1}
\begin{center}
{\renewcommand{\arraystretch}{1.2} 
\hspace{-2.2cm}
	\noindent\makebox[\textwidth]{
     \begin{tabular}{  l p{12cm} }
     \textbf{Defect:} 						& It is a generic term which refers to either a fault or a failure, where fault pertains to the cause and failure to the effect. \\
     \textbf{Fault:} 						& It is a flaw in the system which may result in lack of capability or failure of the system.\\
     \textbf{Failure:} 						& It is a malfunction of a system within the specified requirements or inappropriate performance.\\
     \textbf{Error:} 						& It is a mistake or omission on the part of humans resulting in faulty software. \\
     \textbf{Isolation:} 						& It is to identify the basic cause of failure in the SUT.\\
     \textbf{Detection:} 					& It checks the difference between the observed behaviour of a program and the expected behaviour.\\
     \textbf{Test case:} 					& It is an artefact which delineates the input, action and expected output corresponding to that input.\\
     \textbf{Test specification:} 				& It contains requirement which should be satisfied by test cases.\\
     \textbf{Test execution:} 				& It is the process of running test cases.\\
     \textbf{Test Plan:} 					& It is a document which defines the goal, scope, method, resources and time schedule of testing.\\
     \textbf{Test suite:} 					& It is a set of one or more test cases.\\
     \textbf{Test strategy:}   	  			& It is a method which defines the procedure of testing distinctive characteristics of a program.\\
     \textbf{Validation:}  					& It is a process to assess the software in order to assure that it satisfies the customer requirements. \\
     \textbf{Verification:}  					& It is a process of checking the software to verify that it works correctly.\\
     \textbf{Correctness:} 					& It is the ability of a software to perform as expected by its specification.\\
     \textbf{Test oracle:}					& It is a source containing expected results for comparison with the actual results of the SUT.\\	
     \textbf{Robustness:}  					& It is the ability of a software to handle situations not defined by its specification.\\
     \textbf{Efficiency:} 					& It is the number of defects discovered by a technique per unit time.\\
     \end{tabular}}
     }
\end{center}
 
% It is a table for the next page glossaries.

 \begin{center}
 {\renewcommand{\arraystretch}{1.2} 
\hspace{-2.2cm}
	\noindent\makebox[\textwidth]{
     \begin{tabular}{  l p{12cm} }
     \textbf{Effectiveness:} 					& It is the number of defects discovered in a SUT by a testing technique.\\
%     \textbf{Validation:}  					& It is a process to assess the software in order to assure that it satisfies the customer requirements. \\
%     \textbf{Verification:}  					& It is a process of checking the software to verify that it works correctly.\\
     \end{tabular}}
     }
\end{center}



\end{glossary1}