\chapter*{Glossary}
\addcontentsline{toc}{chapter}{Glossary}
\label{chap:glossary1}

%\scriptsize
 \begin{center}
{\renewcommand{\arraystretch}{1.2} 
\hspace{-2.2cm}
	\noindent\makebox[\textwidth]{
     \begin{tabular}{  l p{12cm} }
	\textbf{Branch:}						& Conditional transfer of control from one statement to another in the code.\\ 
	\textbf{Correctness:} 					& Ability of software to perform according to the given specifications.\\
       \textbf{Dead code}					& Unreachable code in program that cannot be executed.\\
       \textbf{Defect:} 						& Generic term referring to fault or failure.\\      
       \textbf{Detection:} 					& Difference between observed and expected behaviour of program.\\     
	\textbf{Effectiveness:} 				& Number of defects discovered in the program by a testing technique.\\
       \textbf{Efficiency:} 					& Number of defects discovered per unit time by a testing technique.\\
       \textbf{Error:} 						& Mistake or omission in the software. \\
       \textbf{Failure:} 						& Malfunction of a software.\\    
       \textbf{Fault:} 						& Any flaw in the software resulting in lack of capability or failure.\\
       \textbf{Instrumentation:}				& Insertion of additional code in the program for analysis.\\
       \textbf{Invariant:} 					& A condition which must hold true during program execution.\\
       \textbf{Isolation:} 				 	& Identification of the basic cause of failure in software.\\
       \textbf{Path:}						& A sequence of executable statements from entry to exit point in software.\\
       \textbf{Postcondition:}				& A condition which must be true after execution.\\
       \textbf{Precondition:}					& A condition which must be true before execution.\\
    	\textbf{Robustness:}  					& The degree to which a system can function correctly with invalid inputs.\\
      \textbf{Test case:} 					& An artefact that delineates the input, action and expected output.\\    
     \textbf{Test coverage:}					& Number of instructions exercised divided by total number of instructions expressed in percentage.\\ 
     
     \end{tabular}}
     }
\end{center}
 
% It is a table for the next page glossaries.

 \begin{center}
 {\renewcommand{\arraystretch}{1.2} 
\hspace{-2.2cm}
	\noindent\makebox[\textwidth]{
     \begin{tabular}{  l p{12cm} }
     


    \textbf{Test execution:} 				& The process of executing test case.\\
 	\textbf{Test oracle:}					& A mechanism used to determine whether a test has passed or failed.\\
   	\textbf{Test Plan:} 					& A document which defines the goal, scope, method, resources and time schedule of testing.\\
       \textbf{Test specification:} 			& The requirements which should be satisfied by test cases.\\
      \textbf{Test strategy:}   	  			& The method which defines the procedure of testing of a program.\\     	
      \textbf{Test suite:} 					& A set of one or more test cases.\\
     \textbf{Validation:}  					& Assessment of software to ensure satisfaction of customer requirements. \\
     \textbf{Verification:}  					& Checking of software for verification of working properly. \\

     

     


     
%     \textbf{Validation:}  					& It is a process to assess the software in order to assure that it satisfies the customer requirements. \\
%     \textbf{Verification:}  				& It is a process of checking the software to verify that it works correctly.\\
     \end{tabular}}
     }
\end{center}
