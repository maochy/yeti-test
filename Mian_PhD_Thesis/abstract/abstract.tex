\begin{abstract}

%Software testing is the process of evaluating the quality of a software or its component.The thesis presents new techniques for improving the effectiveness of automated random testing, evaluates the efficiency of these techniques and proposes directions for future work.
%, is a well used approach for detecting software failures. Testing involves generation and execution of test inputs and evaluation of results for correctness either manually or by automatic means. Automated software testing save time and human effort involved in manual testing. There are two major challenges in software testing: the generation of appropriate test inputs and evaluation of the test results. This thesis addresses both issues.

%Exhaustive testing is not feasible in most cases and a test strategy is often used to select a small subset of the inputs for testing of software. Selection of adequate test strategy is crucial for better test performance because the chances of finding failures increases if the selected subset of data effectively represents the whole input domain. 


The ever increasing reliance on software-intensive systems is driving research to discover software faults more effectively and more efficiently. Despite intensive research, very few approaches have studied and used knowledge about fault domains to improve the testing or the feedback given to developers. The present thesis addresses this shortcoming: it leverages fault co-localization in a new random testing strategy called Dirt Spot Sweeping Random (DSSR), and it presents two new strategies: Automated Discovery of Failure Domain (ADFD) and Automated Discovery of Failure Domain$^+$ (ADFD$^+$). These improve the feedback given to developers by deducing more information about the failure domain than single failure in an automated way.
%Testing a software with all permutations and combinations of inputs is not feasible because of infinite possible scenarios. Alternatively, a strategy is used to select a small subset of inputs for testing software. Random strategy is a viable option to generate comparatively cheap test inputs without too much intellectual and computational efforts. However, arbitrarily generating test inputs without any help from available information may be effective in some cases but generally results in vague or unnecessary test inputs.  
%The strategy which uses a truly representative subset of the input domain increases the chances of detecting higher number of failure in the software.  
%The thesis addresses the issues and presents three new automated random testing strategies developed by manipulating the patterns of failure domains within the input domain. 
%The strategies have been experimentally evaluated for effectiveness and efficiency. The characteristics of failure domains and their influence on the performance of the test strategies have been examined. 
The DSSR strategy adds the value causing the failure and its neighbouring values to the list of interesting values for exploring the underlying failure domain. The comparative evaluation showed significantly better performance of DSSR over Random and Random$^+$ strategies. The ADFD strategy finds failures and failure domains and presents the pass and fail domains in graphical form. The results obtained by evaluating error-seeded numerical programs indicated highly effective performance of the ADFD strategy. The ADFD$^+$ strategy is an extended version of ADFD strategy with respect to algorithm and graphical presentation of failure domains. In comparison with Randoop, ADFD$^+$ strategy successfully detected all failures and failure domains while Randoop identified individual failures but could not detect failure domains. 
%To determine the precision of identifying failure domains by ADFD and ADFD+, Daikon was integrated in the two techniques and extensive experimental analyses of real world Java projects contained in Qualitas Corpus were performed. The results obtained were analysed and cross-checked by manual testing. 
The ADFD and ADFD$^+$ techniques were enhanced by integration of the automatic invariant detector Daikon, and the precision of identifying failure domains was determined through extensive experimental evaluation of real world Java projects contained in a database, namely Qualitas Corpus. The analyses of results, cross-checked by manual testing indicated that ADFD and ADFD$^+$ techniques are highly effective in providing assistance but are not an alternative to manual testing with the limited available resources.


%These results provide a thorough understanding of automated random testing and leads to several researchable areas indicated in the thesis.


%The chapter evaluates the precision of identifying failure domains by the enhanced ADFD and ADFD+ techniques integrated with the automatic tool Daikon. Extensive experimental analysis of real world Java projects contained in Qualitas Corpus were performed. The results obtained were analysed and cross-checked with the results of manual testing. The results reveal that the two techniques can effectively identify and present all types of failure domains (graphically by JFreeChart and as invariants by Daikon) to a certain level of precision. It is also evident that the level of precision of identifying failure domain can be further increased graphically and invariantly by increasing the range value in the two techniques. The analysis revealed that the strip failure domain having large size and low complexity are quickly identified by the automated techniques whereas the point and block failure domains having small size and higher complexity are difficultly identified by the auto- mated and manual techniques. Based on the results, it can also be stated that automated techniques (ADFD and ADFD+) can be highly effective in providing assistance to manual testing but are not an alternative to the manual testing.











%To find the comparative effectiveness of ADFD and ADFD+ strategies, both techniques were integrated with automated invariant detector Daikon. Extensive experimental analysis of Java projects contained in Qualitas Corpus was carried out in comparison with manual technique. The results revealed 



%Evaluation of the precision of identifying failure domains by ADFD and ADFD+. For the purpose of comparative analysis, Daikon has been integrated in the two techniques and extensive experimental analyses of real world Java projects contained in Qualitas Corpus are performed. The results obtained are analysed and cross-checked with the results of Manual testing. The impact of nature, location, size, type and complexity of failure domains on the testing techniques are reflected.



%This thesis presents new automated random testing strategies developed by manipulating the patterns of failure domains within the input domain. The strategies have been experimentally evaluated for effectiveness and efficiency. The characteristics of failure domains and their influence on the performance of the test strategies has been examined. A brief introduction is given in Chapter~\ref{Introduction1} which is followed by a detailed review of the relevant literature in Chapter~\ref{chap:softwareTesting}. Chapter~\ref{chap:yeti_3} includes a thorough review of YETI, that has been used as a platform to host the strategies and conduct the experiments. Chapter~\ref{chap:DSSR} describes the DSSR strategy based on the dirt spot sweeping phenomenon that adds the value causing the failure and its neighbouring values to the list if interesting values for exploring the underlying failure domain. The comparative evaluation showed significantly better performance of DSSR over R and R$^+$ strategies. Chapter~\ref{chap:ADFD} presents the Automated Discovery of Failure Domain (ADFD) strategy developed with the ability to find failure, failure domains and provides visualisation of pass and fail domain. The experimental results by applying ADFD strategy to error-seeded programs indicate that the strategy is highly effective in identifying the failure domains. Chapter~\ref{chap:ADFD+} presents the Automated Discovery of Failure Domain$^+$ (ADFD$^+$) strategy. It is an upgraded version of ADFD strategy with respect to the algorithm and graphical representation of failure domains. The ADFD$^+$ strategy compared with Randoop under identical conditions successfully detected all failure domains as against Randoop, which identified individual failures but was unable to detect the failure domains. {Chapter~\ref{chap:Evaluation}} presents extensive experimental analysis of Java projects contained in Qualitas Corpus to find the effectiveness of the two automated techniques (ADFD and ADFD$^+$) in comparison with manual technique. The results revealed significance of the two techniques and also provide an insight into the types, frequencies, nature of failure domains and their effect on the testing techniques in production software. {Chapter~\ref{chap:conclusions_8}} includes conclusions, contributions and the lessons learned. Finally, {Chapter~\ref{chap:futureWork}} highlights the opportunities for future work, challenges and likely solutions.

%The types, frequencies, nature of failure domains and their effect on the testing techniques in production software were explored. {Chapter~\ref{chap:conclusions_8}} includes conclusions, contributions and the lessons learned. Finally, {Chapter~\ref{chap:futureWork}} highlights the opportunities for future work, challenges and likely solutions.


%Software is an important and essential component of computer system without which no task can be accomplished. Software testing, the process of evaluating the correctness and quality of a software or its component, is the most widely adapted method for detecting software failures. For program testing, test inputs are generated, executed and the results are evaluated for correctness. Automated software testing is performed to save time and human effort involved in manual testing. The two major challenges in automated testing i.e. the generation of appropriate test inputs and evaluation of the test results need to be addressed.

%The present work is an addition to the literature aiming at reducing the overall cost of software testing by devising new, improved and  effective automated software testing techniques based on random strategy. The first new technique named as Dirt Spot Sweeping Random (DSSR) strategy was developed on the assumption that unique failures reside in contiguous block strips. When a failure is identified, the DSSR strategy selects the neighbouring input values except duplicate values for the subsequent tests. The selected values sweep around the identified failure, leading to the discovery of new failures in the vicinity. This results in quick and efficient identification of faults in Software Under Test (SUT). The second technique named as, Automated Discovery of Failure Domain (ADFD) was developed with the capability to find faults as well as the fault domains in a given SUT and provides visualization of the identified pass and fail domains within a specified range in the form of a chart. The new technique is highly effective in testing and debugging and provides an easy to understand test report in the visualized form. The third new technique proposed in the research study is, Invariant Guided Random$^+$ Strategy (IGRS) which is an extended form of Random$^+$ strategy guided by software invariants. In this technique, Invariants from the given SUT are automatically collected by Daikon tool, filtered through DynComp and annotated in the source code as assertions. (The experiments are in progress, the results obtained will be compared with the DSSR, Random and Random$^+$ strategies and the findings will be included in the thesis and abbreviated in the abstract as soon as possible.)

\end{abstract}