\begin{abstract}
Chapter 4 describes newly proposed more efficient random testing technique known as Dirt Spot Sweeping Random (DSSR) strategy, which is based on sweeping of fault clusters in the input domain. The experimental study confirms that DSSR strategy is significantly better than R and R+ strategies. Finally the benefits and drawbacks of the DSSR strategy are discussed. \\

Chapter 5 presents the newly developed Automated Discovery of Fault Domains (ADFD) strategy, which focuses on dynamically finding the faults and domains along with their graphical representation. It is shown that the presence of fault domains across the input domain, which have not been so far identified, can in fact be identified and graphically represented using ADFD strategy. \\

Chapter 6 presents the newly developed Invariant Guided Random+ Strategy (IGRS) developed with the focus on quick identification of faults and increase in code coverage with the help of assertions. It uses Daikon, a tool to generate likely invariants, to identify and incorporate invariants in the SUT code which serves as contracts to filter any vague test cases. \\

Chapter 7 gives an overall conclusion of the thesis. The main conclusion is that the proposed techniques DSSR and IGRS represents a significant improvement over pure random testing under similar condition. The ADFD strategy not only identify but correctly plot the discovered fault domains in one and two dimensional programs in a specified range. \\
\end{abstract}