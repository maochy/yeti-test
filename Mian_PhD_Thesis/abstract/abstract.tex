\begin{abstract}

This thesis presents new techniques for improving the effectiveness of automated random testing, evaluates the efficiency of these techniques, and proposes directions for future research.

Software testing is currently the most widely used method for detecting software failures. When
testing a program, developers need to generate test inputs for the program, run these test inputs on the program, and check the test execution for correctness. It has been well recognized that software testing is quite expensive, and automated software testing is important for reducing the laborious human effort in testing. There are at least two major technical challenges in automated testing: the generation of sufficient test inputs and the checking of the test execution for correctness. Program specifications can be valuable in addressing these two challenges. 

This thesis presents a framework for improving effectiveness of automated testing in the
absence of specifications. The framework supports a set of related techniques. First, Dirt Spot Sweeping Random (DSSR) strategy was developed. It is based on the assumption that unique failures reside in contiguous blocks and stripes. When a failure is identified, the DSSR strategy selects neighbouring values for the subsequent tests. Resultantly, selected values sweep around the failure, leading to the discovery of new failures in the vicinity. Second, Automated Discovery of Failure Domain (ADFD)” was developed with the ability to find the faults as well as the fault domains in a given SUT and provides visualization of the identified pass and fail domains in the form of a chart. Third, IGRS is an extended form of Random+ strategy guided by software invariants. Invariants from the given SUT are collected by Daikon— an automated invariant detector for reporting likely invariants, filtered using DynComp and annotated them in to the source code as assertions. 

The framework has been implemented and empirical results have shown that the developed techniques within the framework improve the effectiveness of automated testing by detecting high percentage of redundant tests among test inputs generated by existing tools, generating non redundant test inputs to achieve high structural coverage, reducing inspection efforts for detecting problems in the program, and exposing more behavioural differences during regression testing.


\end{abstract}