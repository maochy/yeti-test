\begin{abstract}

%Software testing is the process of evaluating the quality of a software or its component.The thesis presents new techniques for improving the effectiveness of automated random testing, evaluates the efficiency of these techniques and proposes directions for future work.
%, is a well used approach for detecting software failures. Testing involves generation and execution of test inputs and evaluation of results for correctness either manually or by automatic means. Automated software testing save time and human effort involved in manual testing. There are two major challenges in software testing: the generation of appropriate test inputs and evaluation of the test results. This thesis addresses both issues.

Exhaustive testing is not feasible in most cases and a test strategy is often used to select a small subset of the inputs for testing the software. Selection of adequate test strategy is crucial for better test performance because the chances of finding failures increases if the selected subset of data effectively represents the whole input domain. 

In this thesis, we presents new techniques for improving the effectiveness of automated random testing, evaluates the efficiency of these techniques and proposes directions for future work.

The first technique, Dirt Spot Sweeping Random (DSSR) strategy is developed on the assumption that unique failures reside in contiguous block and strips. It starts by testing the code at random. When a failure is identified, the DSSR strategy selects the neighbouring input values for the subsequent tests. The selected values sweep around the identified failure leading to the discovery of new failures in the vicinity. This results in quick and efficient identification of new failures in SUT.

The second technique, Automated Discovery of Failure Domain (ADFD) is developed with the capability to find failure and the failure-domains in a given SUT and provides visualization of the identified pass and fail domains within a specified range in the form of a chart. The new technique is highly effective in testing and debugging and provides an easy to understand test report in the visualized form.

The third technique, Automated Discovery of Failure Domain+ (ADFD+) is an upgraded version of ADFD technique with respect to algorithm and graphical representation of failure domains. To find the effectiveness of ADFD+, it was compared with Randoop using error seeded programs. The ADFD+ correctly pointed out all the seeded failure domains while Randoop identified individual failures but was unable to discover the failure domains.


%Software is an important and essential component of computer system without which no task can be accomplished. Software testing, the process of evaluating the correctness and quality of a software or its component, is the most widely adapted method for detecting software failures. For program testing, test inputs are generated, executed and the results are evaluated for correctness. Automated software testing is performed to save time and human effort involved in manual testing. The two major challenges in automated testing i.e. the generation of appropriate test inputs and evaluation of the test results need to be addressed.

%The present work is an addition to the literature aiming at reducing the overall cost of software testing by devising new, improved and  effective automated software testing techniques based on random strategy. The first new technique named as Dirt Spot Sweeping Random (DSSR) strategy was developed on the assumption that unique failures reside in contiguous block strips. When a failure is identified, the DSSR strategy selects the neighbouring input values except duplicate values for the subsequent tests. The selected values sweep around the identified failure, leading to the discovery of new failures in the vicinity. This results in quick and efficient identification of faults in Software Under Test (SUT). The second technique named as, Automated Discovery of Failure Domain (ADFD) was developed with the capability to find faults as well as the fault domains in a given SUT and provides visualization of the identified pass and fail domains within a specified range in the form of a chart. The new technique is highly effective in testing and debugging and provides an easy to understand test report in the visualized form. The third new technique proposed in the research study is, Invariant Guided Random+ Strategy (IGRS) which is an extended form of Random+ strategy guided by software invariants. In this technique, Invariants from the given SUT are automatically collected by Daikon tool, filtered through DynComp and annotated in the source code as assertions. (The experiments are in progress, the results obtained will be compared with the DSSR, Random and Random+ strategies and the findings will be included in the thesis and abbreviated in the abstract as soon as possible.)

\end{abstract}