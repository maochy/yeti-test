\begin{abstract}

The thesis presents new techniques for improving the effectiveness of automated random testing, evaluates the efficiency of these techniques and proposes directions for future work.

Software testing, the process of evaluating the correctness and quality of a software or its component, is a well used approach for detecting software failures. Testing involves generation and execution of test inputs and evaluation of results for correctness either manually or by automatic means. Automated software testing save time and human effort involved in manual testing. Two major challenges in automated testing i.e. the generation of appropriate test inputs and evaluation of the test results have been addressed in the thesis.

To address the issue of selecting appropriate input data a new technique named as Dirt Spot Sweeping Random (DSSR) strategy is developed on the assumption that unique failures reside in contiguous block strips. When a failure is identified, the DSSR strategy selects the neighbouring input values except duplicate values for the subsequent tests. The selected values sweep around the identified failure, leading to the discovery of new failures in the vicinity. This results in quick and efficient identification of faults in Software Under Test (SUT).

To address the issue of evaluating test results a new technique named as, Automated Discovery of Failure Domain (ADFD) is developed with the capability to find faults as well as the fault domains in a given SUT and provides visualization of the identified pass and fail domains within a specified range in the form of a chart. The new technique is highly effective in testing and debugging and provides an easy to understand test report in the visualized form.

The third new technique proposed in the research study is, Invariant Guided Random+ Strategy (IGRS) which is an extended form of Random+ strategy guided by software invariants. In this technique, Invariants from the given SUT are automatically collected by Daikon tool, filtered through DynComp and annotated in the source code as assertions. (The experiments are in progress, the results obtained will be compared with the DSSR, Random and Random+ strategies and the findings will be included in the thesis and abbreviated in the abstract as soon as possible.)


%Software is an important and essential component of computer system without which no task can be accomplished. Software testing, the process of evaluating the correctness and quality of a software or its component, is the most widely adapted method for detecting software failures. For program testing, test inputs are generated, executed and the results are evaluated for correctness. Automated software testing is performed to save time and human effort involved in manual testing. The two major challenges in automated testing i.e. the generation of appropriate test inputs and evaluation of the test results need to be addressed.

%The present work is an addition to the literature aiming at reducing the overall cost of software testing by devising new, improved and  effective automated software testing techniques based on random strategy. The first new technique named as Dirt Spot Sweeping Random (DSSR) strategy was developed on the assumption that unique failures reside in contiguous block strips. When a failure is identified, the DSSR strategy selects the neighbouring input values except duplicate values for the subsequent tests. The selected values sweep around the identified failure, leading to the discovery of new failures in the vicinity. This results in quick and efficient identification of faults in Software Under Test (SUT). The second technique named as, Automated Discovery of Failure Domain (ADFD) was developed with the capability to find faults as well as the fault domains in a given SUT and provides visualization of the identified pass and fail domains within a specified range in the form of a chart. The new technique is highly effective in testing and debugging and provides an easy to understand test report in the visualized form. The third new technique proposed in the research study is, Invariant Guided Random+ Strategy (IGRS) which is an extended form of Random+ strategy guided by software invariants. In this technique, Invariants from the given SUT are automatically collected by Daikon tool, filtered through DynComp and annotated in the source code as assertions. (The experiments are in progress, the results obtained will be compared with the DSSR, Random and Random+ strategies and the findings will be included in the thesis and abbreviated in the abstract as soon as possible.)

\end{abstract}